% Generated by Sphinx.
\def\sphinxdocclass{report}
\newif\ifsphinxKeepOldNames \sphinxKeepOldNamestrue
\documentclass[letterpaper,10pt,english]{sphinxmanual}
\usepackage{iftex}

\ifPDFTeX
  \usepackage[utf8]{inputenc}
\fi
\ifdefined\DeclareUnicodeCharacter
  \DeclareUnicodeCharacter{00A0}{\nobreakspace}
\fi
\usepackage{cmap}
\usepackage[T1]{fontenc}
\usepackage{amsmath,amssymb,amstext}
\usepackage{babel}
\usepackage{times}
\usepackage[Bjarne]{fncychap}
\usepackage{longtable}
\usepackage{sphinx}
\usepackage{multirow}
\usepackage{eqparbox}


\addto\captionsenglish{\renewcommand{\figurename}{Fig.\@ }}
\addto\captionsenglish{\renewcommand{\tablename}{Table }}
\SetupFloatingEnvironment{literal-block}{name=Listing }

\addto\extrasenglish{\def\pageautorefname{page}}




\title{IOR Analysis Documentation}
\date{Apr 27, 2018}
\release{0.1.1}
\author{John C Harrington}
\newcommand{\sphinxlogo}{}
\renewcommand{\releasename}{Release}
\makeindex

\makeatletter
\def\PYG@reset{\let\PYG@it=\relax \let\PYG@bf=\relax%
    \let\PYG@ul=\relax \let\PYG@tc=\relax%
    \let\PYG@bc=\relax \let\PYG@ff=\relax}
\def\PYG@tok#1{\csname PYG@tok@#1\endcsname}
\def\PYG@toks#1+{\ifx\relax#1\empty\else%
    \PYG@tok{#1}\expandafter\PYG@toks\fi}
\def\PYG@do#1{\PYG@bc{\PYG@tc{\PYG@ul{%
    \PYG@it{\PYG@bf{\PYG@ff{#1}}}}}}}
\def\PYG#1#2{\PYG@reset\PYG@toks#1+\relax+\PYG@do{#2}}

\expandafter\def\csname PYG@tok@sc\endcsname{\def\PYG@tc##1{\textcolor[rgb]{0.25,0.44,0.63}{##1}}}
\expandafter\def\csname PYG@tok@gi\endcsname{\def\PYG@tc##1{\textcolor[rgb]{0.00,0.63,0.00}{##1}}}
\expandafter\def\csname PYG@tok@s\endcsname{\def\PYG@tc##1{\textcolor[rgb]{0.25,0.44,0.63}{##1}}}
\expandafter\def\csname PYG@tok@nv\endcsname{\def\PYG@tc##1{\textcolor[rgb]{0.73,0.38,0.84}{##1}}}
\expandafter\def\csname PYG@tok@gh\endcsname{\let\PYG@bf=\textbf\def\PYG@tc##1{\textcolor[rgb]{0.00,0.00,0.50}{##1}}}
\expandafter\def\csname PYG@tok@nt\endcsname{\let\PYG@bf=\textbf\def\PYG@tc##1{\textcolor[rgb]{0.02,0.16,0.45}{##1}}}
\expandafter\def\csname PYG@tok@se\endcsname{\let\PYG@bf=\textbf\def\PYG@tc##1{\textcolor[rgb]{0.25,0.44,0.63}{##1}}}
\expandafter\def\csname PYG@tok@nn\endcsname{\let\PYG@bf=\textbf\def\PYG@tc##1{\textcolor[rgb]{0.05,0.52,0.71}{##1}}}
\expandafter\def\csname PYG@tok@nc\endcsname{\let\PYG@bf=\textbf\def\PYG@tc##1{\textcolor[rgb]{0.05,0.52,0.71}{##1}}}
\expandafter\def\csname PYG@tok@mh\endcsname{\def\PYG@tc##1{\textcolor[rgb]{0.13,0.50,0.31}{##1}}}
\expandafter\def\csname PYG@tok@err\endcsname{\def\PYG@bc##1{\setlength{\fboxsep}{0pt}\fcolorbox[rgb]{1.00,0.00,0.00}{1,1,1}{\strut ##1}}}
\expandafter\def\csname PYG@tok@cs\endcsname{\def\PYG@tc##1{\textcolor[rgb]{0.25,0.50,0.56}{##1}}\def\PYG@bc##1{\setlength{\fboxsep}{0pt}\colorbox[rgb]{1.00,0.94,0.94}{\strut ##1}}}
\expandafter\def\csname PYG@tok@c1\endcsname{\let\PYG@it=\textit\def\PYG@tc##1{\textcolor[rgb]{0.25,0.50,0.56}{##1}}}
\expandafter\def\csname PYG@tok@kt\endcsname{\def\PYG@tc##1{\textcolor[rgb]{0.56,0.13,0.00}{##1}}}
\expandafter\def\csname PYG@tok@cm\endcsname{\let\PYG@it=\textit\def\PYG@tc##1{\textcolor[rgb]{0.25,0.50,0.56}{##1}}}
\expandafter\def\csname PYG@tok@nd\endcsname{\let\PYG@bf=\textbf\def\PYG@tc##1{\textcolor[rgb]{0.33,0.33,0.33}{##1}}}
\expandafter\def\csname PYG@tok@dl\endcsname{\def\PYG@tc##1{\textcolor[rgb]{0.25,0.44,0.63}{##1}}}
\expandafter\def\csname PYG@tok@ch\endcsname{\let\PYG@it=\textit\def\PYG@tc##1{\textcolor[rgb]{0.25,0.50,0.56}{##1}}}
\expandafter\def\csname PYG@tok@sh\endcsname{\def\PYG@tc##1{\textcolor[rgb]{0.25,0.44,0.63}{##1}}}
\expandafter\def\csname PYG@tok@mo\endcsname{\def\PYG@tc##1{\textcolor[rgb]{0.13,0.50,0.31}{##1}}}
\expandafter\def\csname PYG@tok@ss\endcsname{\def\PYG@tc##1{\textcolor[rgb]{0.32,0.47,0.09}{##1}}}
\expandafter\def\csname PYG@tok@mb\endcsname{\def\PYG@tc##1{\textcolor[rgb]{0.13,0.50,0.31}{##1}}}
\expandafter\def\csname PYG@tok@cp\endcsname{\def\PYG@tc##1{\textcolor[rgb]{0.00,0.44,0.13}{##1}}}
\expandafter\def\csname PYG@tok@kc\endcsname{\let\PYG@bf=\textbf\def\PYG@tc##1{\textcolor[rgb]{0.00,0.44,0.13}{##1}}}
\expandafter\def\csname PYG@tok@gp\endcsname{\let\PYG@bf=\textbf\def\PYG@tc##1{\textcolor[rgb]{0.78,0.36,0.04}{##1}}}
\expandafter\def\csname PYG@tok@k\endcsname{\let\PYG@bf=\textbf\def\PYG@tc##1{\textcolor[rgb]{0.00,0.44,0.13}{##1}}}
\expandafter\def\csname PYG@tok@nl\endcsname{\let\PYG@bf=\textbf\def\PYG@tc##1{\textcolor[rgb]{0.00,0.13,0.44}{##1}}}
\expandafter\def\csname PYG@tok@gt\endcsname{\def\PYG@tc##1{\textcolor[rgb]{0.00,0.27,0.87}{##1}}}
\expandafter\def\csname PYG@tok@sd\endcsname{\let\PYG@it=\textit\def\PYG@tc##1{\textcolor[rgb]{0.25,0.44,0.63}{##1}}}
\expandafter\def\csname PYG@tok@s1\endcsname{\def\PYG@tc##1{\textcolor[rgb]{0.25,0.44,0.63}{##1}}}
\expandafter\def\csname PYG@tok@m\endcsname{\def\PYG@tc##1{\textcolor[rgb]{0.13,0.50,0.31}{##1}}}
\expandafter\def\csname PYG@tok@sr\endcsname{\def\PYG@tc##1{\textcolor[rgb]{0.14,0.33,0.53}{##1}}}
\expandafter\def\csname PYG@tok@nf\endcsname{\def\PYG@tc##1{\textcolor[rgb]{0.02,0.16,0.49}{##1}}}
\expandafter\def\csname PYG@tok@sa\endcsname{\def\PYG@tc##1{\textcolor[rgb]{0.25,0.44,0.63}{##1}}}
\expandafter\def\csname PYG@tok@vg\endcsname{\def\PYG@tc##1{\textcolor[rgb]{0.73,0.38,0.84}{##1}}}
\expandafter\def\csname PYG@tok@mi\endcsname{\def\PYG@tc##1{\textcolor[rgb]{0.13,0.50,0.31}{##1}}}
\expandafter\def\csname PYG@tok@vc\endcsname{\def\PYG@tc##1{\textcolor[rgb]{0.73,0.38,0.84}{##1}}}
\expandafter\def\csname PYG@tok@vm\endcsname{\def\PYG@tc##1{\textcolor[rgb]{0.73,0.38,0.84}{##1}}}
\expandafter\def\csname PYG@tok@c\endcsname{\let\PYG@it=\textit\def\PYG@tc##1{\textcolor[rgb]{0.25,0.50,0.56}{##1}}}
\expandafter\def\csname PYG@tok@si\endcsname{\let\PYG@it=\textit\def\PYG@tc##1{\textcolor[rgb]{0.44,0.63,0.82}{##1}}}
\expandafter\def\csname PYG@tok@nb\endcsname{\def\PYG@tc##1{\textcolor[rgb]{0.00,0.44,0.13}{##1}}}
\expandafter\def\csname PYG@tok@ne\endcsname{\def\PYG@tc##1{\textcolor[rgb]{0.00,0.44,0.13}{##1}}}
\expandafter\def\csname PYG@tok@kn\endcsname{\let\PYG@bf=\textbf\def\PYG@tc##1{\textcolor[rgb]{0.00,0.44,0.13}{##1}}}
\expandafter\def\csname PYG@tok@kr\endcsname{\let\PYG@bf=\textbf\def\PYG@tc##1{\textcolor[rgb]{0.00,0.44,0.13}{##1}}}
\expandafter\def\csname PYG@tok@il\endcsname{\def\PYG@tc##1{\textcolor[rgb]{0.13,0.50,0.31}{##1}}}
\expandafter\def\csname PYG@tok@kd\endcsname{\let\PYG@bf=\textbf\def\PYG@tc##1{\textcolor[rgb]{0.00,0.44,0.13}{##1}}}
\expandafter\def\csname PYG@tok@vi\endcsname{\def\PYG@tc##1{\textcolor[rgb]{0.73,0.38,0.84}{##1}}}
\expandafter\def\csname PYG@tok@bp\endcsname{\def\PYG@tc##1{\textcolor[rgb]{0.00,0.44,0.13}{##1}}}
\expandafter\def\csname PYG@tok@no\endcsname{\def\PYG@tc##1{\textcolor[rgb]{0.38,0.68,0.84}{##1}}}
\expandafter\def\csname PYG@tok@ni\endcsname{\let\PYG@bf=\textbf\def\PYG@tc##1{\textcolor[rgb]{0.84,0.33,0.22}{##1}}}
\expandafter\def\csname PYG@tok@s2\endcsname{\def\PYG@tc##1{\textcolor[rgb]{0.25,0.44,0.63}{##1}}}
\expandafter\def\csname PYG@tok@sx\endcsname{\def\PYG@tc##1{\textcolor[rgb]{0.78,0.36,0.04}{##1}}}
\expandafter\def\csname PYG@tok@na\endcsname{\def\PYG@tc##1{\textcolor[rgb]{0.25,0.44,0.63}{##1}}}
\expandafter\def\csname PYG@tok@fm\endcsname{\def\PYG@tc##1{\textcolor[rgb]{0.02,0.16,0.49}{##1}}}
\expandafter\def\csname PYG@tok@go\endcsname{\def\PYG@tc##1{\textcolor[rgb]{0.20,0.20,0.20}{##1}}}
\expandafter\def\csname PYG@tok@o\endcsname{\def\PYG@tc##1{\textcolor[rgb]{0.40,0.40,0.40}{##1}}}
\expandafter\def\csname PYG@tok@sb\endcsname{\def\PYG@tc##1{\textcolor[rgb]{0.25,0.44,0.63}{##1}}}
\expandafter\def\csname PYG@tok@gs\endcsname{\let\PYG@bf=\textbf}
\expandafter\def\csname PYG@tok@gd\endcsname{\def\PYG@tc##1{\textcolor[rgb]{0.63,0.00,0.00}{##1}}}
\expandafter\def\csname PYG@tok@gu\endcsname{\let\PYG@bf=\textbf\def\PYG@tc##1{\textcolor[rgb]{0.50,0.00,0.50}{##1}}}
\expandafter\def\csname PYG@tok@cpf\endcsname{\let\PYG@it=\textit\def\PYG@tc##1{\textcolor[rgb]{0.25,0.50,0.56}{##1}}}
\expandafter\def\csname PYG@tok@mf\endcsname{\def\PYG@tc##1{\textcolor[rgb]{0.13,0.50,0.31}{##1}}}
\expandafter\def\csname PYG@tok@ow\endcsname{\let\PYG@bf=\textbf\def\PYG@tc##1{\textcolor[rgb]{0.00,0.44,0.13}{##1}}}
\expandafter\def\csname PYG@tok@gr\endcsname{\def\PYG@tc##1{\textcolor[rgb]{1.00,0.00,0.00}{##1}}}
\expandafter\def\csname PYG@tok@ge\endcsname{\let\PYG@it=\textit}
\expandafter\def\csname PYG@tok@w\endcsname{\def\PYG@tc##1{\textcolor[rgb]{0.73,0.73,0.73}{##1}}}
\expandafter\def\csname PYG@tok@kp\endcsname{\def\PYG@tc##1{\textcolor[rgb]{0.00,0.44,0.13}{##1}}}

\def\PYGZbs{\char`\\}
\def\PYGZus{\char`\_}
\def\PYGZob{\char`\{}
\def\PYGZcb{\char`\}}
\def\PYGZca{\char`\^}
\def\PYGZam{\char`\&}
\def\PYGZlt{\char`\<}
\def\PYGZgt{\char`\>}
\def\PYGZsh{\char`\#}
\def\PYGZpc{\char`\%}
\def\PYGZdl{\char`\$}
\def\PYGZhy{\char`\-}
\def\PYGZsq{\char`\'}
\def\PYGZdq{\char`\"}
\def\PYGZti{\char`\~}
% for compatibility with earlier versions
\def\PYGZat{@}
\def\PYGZlb{[}
\def\PYGZrb{]}
\makeatother

\renewcommand\PYGZsq{\textquotesingle}

\begin{document}

\maketitle
\tableofcontents
\phantomsection\label{index::doc}

\index{IORAnalysis (module)}
This module provides tools for the analysis of IOR era yachts,
using VPP files for each yacht and values from it's rating
certificate.

It allows changes to be made to the yacht rigging, which can
then have it's new rating calculated and it's performance
tested in the VPP.
\index{IOR (class in IORAnalysis)}

\begin{fulllineitems}
\phantomsection\label{index:IORAnalysis.IOR}\pysiglinewithargsret{\sphinxstrong{class }\sphinxcode{IORAnalysis.}\sphinxbfcode{IOR}}{\emph{s, cert=\{\}, ballastChange={[}0, 0{]}}}{}
Calculates IOR Rating for Defined Yacht.
Default values are for Indulgence.

Calculates IOR rating from certificate values.
Provides access to all calculated parameters from the IOR
certificate.
The class performs all calculations required to achieve a
correct SC and CGF for a sloop rig such as Indulgence. It
does not contain any calculations regarding a mizzen or other
unusual sails, or any calculations based on hull measurements.
\begin{quote}\begin{description}
\item[{Parameters}] \leavevmode\begin{itemize}
\item {} 
\textbf{\texttt{cert}} (\sphinxcode{dict}) -- Dictionary containing certificate values for the yacht.

\item {} 
\textbf{\texttt{ballastChange}} (\sphinxcode{list}) -- List containing {[}Ballast Amount, Distance  Moved{]}.
Direction follows standard ship conventions, with
upwards being positive.

\end{itemize}

\end{description}\end{quote}


\begin{fulllineitems}
\pysigline{\sphinxbfcode{Certificate~Values}}
\sphinxcode{floats} -- All certificate values are available as attributes

\end{fulllineitems}

\index{Rating (IORAnalysis.IOR attribute)}

\begin{fulllineitems}
\phantomsection\label{index:IORAnalysis.IOR.Rating}\pysigline{\sphinxbfcode{Rating}}
\sphinxcode{float} -- Rating in it's official form, rounded to 1 d.p.

\end{fulllineitems}

\index{ballastChange (IORAnalysis.IOR attribute)}

\begin{fulllineitems}
\phantomsection\label{index:IORAnalysis.IOR.ballastChange}\pysigline{\sphinxbfcode{ballastChange}}
\sphinxcode{list} -- List containing difference from basis vessel ballast
position in the format {[}Ballast Amount, Distance  Moved{]}.
Direction follows standard ship conventions, with
upwards being positive.

\end{fulllineitems}

\index{actualCGF (IORAnalysis.IOR attribute)}

\begin{fulllineitems}
\phantomsection\label{index:IORAnalysis.IOR.actualCGF}\pysigline{\sphinxbfcode{actualCGF}}
\sphinxcode{float} -- Calculated CGF value. Differs from certificate CGF only if
less than the minimum value of 0.9860.

\end{fulllineitems}

\index{Calc() (IORAnalysis.IOR method)}

\begin{fulllineitems}
\phantomsection\label{index:IORAnalysis.IOR.Calc}\pysiglinewithargsret{\sphinxbfcode{Calc}}{\emph{s}}{}
Calculates final rated length.

Runs calculations for CGF and SC, then calculates
MR, R, Rating and TCF

\end{fulllineitems}

\index{ReqRMChange() (IORAnalysis.IOR method)}

\begin{fulllineitems}
\phantomsection\label{index:IORAnalysis.IOR.ReqRMChange}\pysiglinewithargsret{\sphinxbfcode{ReqRMChange}}{\emph{s}, \emph{reqRating: float}}{{ $\rightarrow$ float}}
Calculates the RM change required to acheive a certain rating.
\begin{quote}\begin{description}
\item[{Parameters}] \leavevmode
\textbf{\texttt{reqRating}} (\sphinxcode{float}) -- The desired rating for which RM change should be calculated, to 1 d.p.

\item[{Returns}] \leavevmode
The required change in righting moment to achieve reqRating,
given in lb feet.

\item[{Return type}] \leavevmode
\sphinxcode{float}

\end{description}\end{quote}

\end{fulllineitems}

\index{updateCert() (IORAnalysis.IOR method)}

\begin{fulllineitems}
\phantomsection\label{index:IORAnalysis.IOR.updateCert}\pysiglinewithargsret{\sphinxbfcode{updateCert}}{\emph{s}, \emph{changes: list}}{}
Updates certificate values stored in attributes
\begin{quote}\begin{description}
\item[{Parameters}] \leavevmode
\textbf{\texttt{changes}} (\sphinxcode{list}) -- \begin{itemize}
\item {} \begin{description}
\item[{\sphinxcode{dict}}] \leavevmode
Dictionary containing changes to certificate values for
the yacht.

\end{description}

\item {} \begin{description}
\item[{\sphinxcode{list}}] \leavevmode
List containing {[}Ballast Amount, Distance  Moved{]}.
Direction follows standard ship conventions, with
upwards being positive.

\end{description}

\end{itemize}


\end{description}\end{quote}

\end{fulllineitems}


\end{fulllineitems}

\index{BaseYacht (class in IORAnalysis)}

\begin{fulllineitems}
\phantomsection\label{index:IORAnalysis.BaseYacht}\pysiglinewithargsret{\sphinxstrong{class }\sphinxcode{IORAnalysis.}\sphinxbfcode{BaseYacht}}{\emph{s}, \emph{cert: dict}, \emph{VPPFile: str}}{}
Define the basic yachts from which changes are made
\index{cert (IORAnalysis.BaseYacht attribute)}

\begin{fulllineitems}
\phantomsection\label{index:IORAnalysis.BaseYacht.cert}\pysigline{\sphinxbfcode{cert}}
\emph{dict} -- Dictionary containing certificate values for the base yacht.

\end{fulllineitems}

\index{VPPFile (IORAnalysis.BaseYacht attribute)}

\begin{fulllineitems}
\phantomsection\label{index:IORAnalysis.BaseYacht.VPPFile}\pysigline{\sphinxbfcode{VPPFile}}
\emph{str} -- String containing path to VPP file for base yacht.

\end{fulllineitems}


\end{fulllineitems}

\index{Yacht (class in IORAnalysis)}

\begin{fulllineitems}
\phantomsection\label{index:IORAnalysis.Yacht}\pysiglinewithargsret{\sphinxstrong{class }\sphinxcode{IORAnalysis.}\sphinxbfcode{Yacht}}{\emph{s, BaseYacht: IORAnalysis.BaseYacht.BaseYacht, changes: list = {[}\{\}, {[}0, 0{]}{]}, h5ID: str = `'}}{}
Generates a yacht ID, runs IOR calculations and
allows retrieval of speed data if it has been calculated.
\index{BaseYacht (IORAnalysis.Yacht attribute)}

\begin{fulllineitems}
\phantomsection\label{index:IORAnalysis.Yacht.BaseYacht}\pysigline{\sphinxbfcode{BaseYacht}}
\sphinxcode{IORAnalysis.BaseYacht.BaseYacht} -- The basis vessel to be modified, provides VPP file and
initial IOR certificate values.

\end{fulllineitems}

\index{IOR (IORAnalysis.Yacht attribute)}

\begin{fulllineitems}
\phantomsection\label{index:IORAnalysis.Yacht.IOR}\pysigline{\sphinxbfcode{IOR}}
\sphinxcode{IORAnalysis.IOR.IOR} -- Instance of the IOR() class. Used to access all rating data.

\end{fulllineitems}

\index{ID (IORAnalysis.Yacht attribute)}

\begin{fulllineitems}
\phantomsection\label{index:IORAnalysis.Yacht.ID}\pysigline{\sphinxbfcode{ID}}
\emph{str} -- Unique identifier string for this yacht, generated from \emph{changes}.

\end{fulllineitems}

\index{changes (IORAnalysis.Yacht attribute)}

\begin{fulllineitems}
\phantomsection\label{index:IORAnalysis.Yacht.changes}\pysigline{\sphinxbfcode{changes}}
\sphinxcode{list} --
\begin{itemize}
\item {} \begin{description}
\item[{\sphinxcode{dict}}] \leavevmode
Dictionary containing changes to certificate values for
the yacht.

\end{description}

\item {} \begin{description}
\item[{\sphinxcode{list}}] \leavevmode
List containing {[}Ballast Amount, Distance  Moved{]}.
Direction follows standard ship conventions, with
upwards being positive.

\end{description}

\end{itemize}

If blank, the yacht is a copy of the basis vessel, but can
then be used to access VPP data if it exists.

\end{fulllineitems}

\index{getSpeed() (IORAnalysis.Yacht method)}

\begin{fulllineitems}
\phantomsection\label{index:IORAnalysis.Yacht.getSpeed}\pysiglinewithargsret{\sphinxbfcode{getSpeed}}{\emph{s}, \emph{ws: int}, \emph{wa: int}}{{ $\rightarrow$ int}}
Returns the time for 1nm for this yacht at the given wind speed
and angle.

\end{fulllineitems}

\index{getSpeedDF() (IORAnalysis.Yacht method)}

\begin{fulllineitems}
\phantomsection\label{index:IORAnalysis.Yacht.getSpeedDF}\pysiglinewithargsret{\sphinxbfcode{getSpeedDF}}{\emph{s}}{{ $\rightarrow$ pandas.core.frame.DataFrame}}
Returns the times for 1nm for this yacht.

\end{fulllineitems}

\index{h5ID (IORAnalysis.Yacht attribute)}

\begin{fulllineitems}
\phantomsection\label{index:IORAnalysis.Yacht.h5ID}\pysigline{\sphinxbfcode{h5ID}}
\sphinxcode{str} -- The identifier for the specific h5 file the VPP data for this
yacht is stored in. Can be used to separate sets of data from
the same basis vessel into different files, for instance one
for each case study.

\end{fulllineitems}


\end{fulllineitems}

\index{WinDes (class in IORAnalysis)}

\begin{fulllineitems}
\phantomsection\label{index:IORAnalysis.WinDes}\pysigline{\sphinxstrong{class }\sphinxcode{IORAnalysis.}\sphinxbfcode{WinDes}}
Wraps WinDes4 class, enforce usage within `with'

This ensures queue will always be either saved
to file after use without continuous file changes.
\begin{quote}\begin{description}
\item[{Returns}] \leavevmode
\textbf{VPP} -- The VPP object.

\item[{Return type}] \leavevmode
{\hyperref[index:IORAnalysis.WinDes.WinDes4]{\sphinxcrossref{\sphinxcode{WinDes4()}}}}

\end{description}\end{quote}
\paragraph{Example}

\begin{Verbatim}[commandchars=\\\{\}]
\PYG{g+gp}{\PYGZgt{}\PYGZgt{}\PYGZgt{} }\PYG{k}{with} \PYG{n}{WinDes}\PYG{p}{(}\PYG{p}{)} \PYG{k}{as} \PYG{n}{VPP}\PYG{p}{:}
\PYG{g+go}{        queue(yacht)}
\end{Verbatim}
\index{WinDes.WinDes4 (class in IORAnalysis)}

\begin{fulllineitems}
\phantomsection\label{index:IORAnalysis.WinDes.WinDes4}\pysiglinewithargsret{\sphinxstrong{class }\sphinxbfcode{WinDes4}}{\emph{s}, \emph{yachtName='Indulgence'}}{}
Run WinDesign4 through PyAutoIt.

Allows queueing of yachts for
analysis when WinDesign is not available. Yachts are stored in
HDF5 database with name format:
``SailVar1-SailVal1.SailVarN-SailValN\_BallastWeight-BallastDistance''
\index{queue() (IORAnalysis.WinDes.WinDes4 method)}

\begin{fulllineitems}
\phantomsection\label{index:IORAnalysis.WinDes.WinDes4.queue}\pysiglinewithargsret{\sphinxbfcode{queue}}{\emph{s}, \emph{Yacht: IORAnalysis.Yacht.Yacht}}{}
Adds a new yacht to the current queue,
if it isn't already there
\begin{quote}\begin{description}
\item[{Parameters}] \leavevmode
\textbf{\texttt{Yacht}} ({\hyperref[index:IORAnalysis.Yacht]{\sphinxcrossref{\sphinxcode{Yacht()}}}}) -- 

\end{description}\end{quote}

\end{fulllineitems}

\index{runQueue() (IORAnalysis.WinDes.WinDes4 method)}

\begin{fulllineitems}
\phantomsection\label{index:IORAnalysis.WinDes.WinDes4.runQueue}\pysiglinewithargsret{\sphinxbfcode{runQueue}}{\emph{s}}{}
Runs VPP for all yachts currently queued

\end{fulllineitems}

\index{saveQueue() (IORAnalysis.WinDes.WinDes4 method)}

\begin{fulllineitems}
\phantomsection\label{index:IORAnalysis.WinDes.WinDes4.saveQueue}\pysiglinewithargsret{\sphinxbfcode{saveQueue}}{\emph{s}}{}
Writes the current queue to a file

\end{fulllineitems}


\end{fulllineitems}


\end{fulllineitems}



\renewcommand{\indexname}{Python Module Index}
\begin{theindex}
\def\bigletter#1{{\Large\sffamily#1}\nopagebreak\vspace{1mm}}
\bigletter{i}
\item {\texttt{IORAnalysis}}, \pageref{index:module-IORAnalysis}
\end{theindex}

\renewcommand{\indexname}{Index}
\printindex
\end{document}
